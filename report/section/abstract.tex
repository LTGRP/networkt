\thispagestyle{plain}
\begin{center}
    \Large
    \textbf{Transnational Diffusion}
    
    \vspace{0.4cm}
    \large
    Across Digital Networks
    
    \vspace{0.4cm}
    \textbf{John Mercouris}
    
    \vspace{0.9cm}
    \textbf{Abstract}
\end{center}

\subsubsection{Motivation}
Transnational Entrepreneurs are conceptualized as individuals embedded
in wider cross-border business networks and socio-political
institutions that shape the attitudes and behaviors of individual
entrepreneurs.

Transnational Entrepreneurship therefore is an ongoing process of
calculated risk taking and foresight in foreign business
venturing. The assumption is that these networks and institutions
provide the necessary strategic infrastructure to enable the success
of Transnational Entrepreneurs. Yet, the question of how the
transnational entrepreneur recieves support from their host and home
country, and how it enhances their risk taking capacities, has
laregely remained unanswered.

\subsubsection{Problem Statement}
The problem is that we have had no way of attempting to characterize
or observe transnational interaction and diffusion faciliated by
Transnational Entrepreneurs. With the advent of digital networks,
particularly public ones, we are able to explore transnational
diffusion and interaction in greater detail and depth.

\subsubsection{Approach}
Our approach involves identifying Transnational Entrepreneurs from
digital networks. Subsequently we attempt to trace the diffusion of a
concept or innovation through their ego centric network, with a focus
on determining whether or not they have facilitated the diffusion
across borders.

\subsubsection{Results}
The results indicate that yes, Transnational Diffusion and interaciton
occurs. More importantly though, Transnational Entrepreneurs act as
facilitators of it. Our next work therefore is to characterize the
frequency and the nature of Transnational Diffusion as compared to
other forms of diffusion across networks and borders.

\subsubsection{Conclusions}
The implication of our findings means that many studies based on the
concept of Transnational Entreprenuership as a phenomenon can be
partially validated as being true within this case. We cannot
generalize our findings to all network types and communication
patterns, but for digital networks we have shown that the pattern does
emerge.







\section{Methodology}
Without directly observing Transnational Entrepreneurs in their own
environments through longitudinal studies, while simultaneously
observing an equally large control group, it is difficult to witness
Transnational Diffusion.

As such as a secondary method of observation was chosen-
Twitter. Through Twitter's public API we are able to get a host of
publicly available data about any given User, and their interactions
within their ego-centric network.

\subsection{Inductive vs Deductive}
Deductive reasoning was chosen because of the characteristics of the
dataset that we have available to us. We first began with a testable
hypothesis - Transnational Diffusion frequency within a Transnational
Entrepreneur's ego-centric network is moderated by their network's
country distribution.

Additionally important was the novelty of deductive reasoning within
this problem space. Prior research focused largely on case studies to
form broader generalizations (inductive), and we wanted to see if we
could validate/invalidate some of the generalized assumptions
discovered in inductive research.

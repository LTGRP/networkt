The breakdown of the implementation section of this paper will break apart every package into a section. Subsequently every single module will be a subsection. At the top of the section will be a brief package definition (what the package does, how it does it) on a high level.
%%%%%%%%%%%%%%%%%%%%%%%%%%%%%%%%%%%%%%%%%%%%%%%%%%%%%%%%%%%%%%%%%%%%%%%%%%%%%%%%
\section{Graph}
The Graph package contains all of the important definitions for the database structure, access, and extraction of data from twitter.
\subsection{The Database Structure}
The database structure is as follows.

%%%%%%%%%%%%%%%%%%%%%%%%%%%%%%%%%%%%%%%%%%%%%%%%%%%%%%%%%%%%%%%%%%%%%%%%%%%%%%%%
\section{Scrapet}
Scrapet is the tool that is responsible for pulling the data from the Twitter API and making the appropriate graphs. Scrapet is the core behind all of the tools in the project. Every single project will end up using a Scrapet dump of data for rendering, machine learning, or any other processes necessary for analysis. This tool is composed of a number of components which will be briefly be introduced below. Follow the introduction and description of components, the high level architecture will be explained.
\subsection{Logger}
The logger is the most important component of the Scrapet system. The logger is an abstract entity that is either fulfilled as a console logger, or as a GUI logger depending on the flavor and execution method of the Scrapet build. The logger is responsible for reporting on the overall progress and the activity of the system.
\subsection{Runner}
The runner is the main entry point of the system. Whether running from the GUI mode, or from the command line mode, Scrapet always begins here. This is where all of the scraping algorithms and functions are organized.
\subsection{Main}
Main is aptly named as the Main entry point into the program. This is where the GUI version of Scrapet begins. Just like the command line program though, the true entry point of execution is in Runner. During execution of the scraping process, scrapet launches 'Runner' as a thread.
\section{Graph}
The graph package contains all of the 
\subsection{Filter Node}
\subsection{Graph}
\subsection{Model}
There is a time and place for results, and placeholders. This is it.\cite{latexcompanion}

%%%%%%%%%%%%%%%%%%%%%%%%%%%%%%%%%%%%%%%%%%%%%%%%%%%%%%%%%%%%%%%%%%%%%%%%%%%%%%%%
\section{Networkt}
Networkt description.
\subsection{Node}
Node Description.


\section{Motivation}
Transnational Entrepreneurs are conceptualized as individuals embedded
in wider cross-border business networks and socio-political
institutions that shape the attitudes and behaviors of individual
entrepreneurs.

Transnational Entrepreneurship therefore is an ongoing process of
calculated risk taking and foresight in foreign business
venturing. The assumption is that these networks and institutions
provide the necessary strategic infrastructure to enable the success
of Transnational Entrepreneurs. Yet, the question of how the
Transnational Entrepreneur recieves support from their host and home
country, and how it enhances their risk taking capacities, has
laregely remained unanswered.

\section{Objective}
The problem is that thus far there has been no way of characterizing
or observing transnational diffusion faciliated by Transnational
Entrepreneurs. ``Although networks have been considered crucial to
transnational entrepreneurship (TE), there has been a lack of
theoretical and methodological engagement with social network analysis
in the existing TE research''\cite{Chen.2009} With the advent of
digital networks, particularly public ones, we are able to witness a
firsthand account of transnational diffusion.

\section{Scope}
Our approach involves identifying Transnational Entrepreneurs within
digital networks. Subsequently, we attempt to trace the diffusion of a
concept or innovation through ego-centric network transnational
networks, with a focus on determining whether or not transantionals
facilitated innovation diffusion across borders.

Upon identification of transnational diffusion activities, we attempt
to characterize the degree to which the activities occur, and how they
may or may not be moderated by a transnational's network composition
in regards to socio-ethnic groups.

\section{Outline}
The results indicate that Transnational Diffusion and interaction
occur within digital networks. More importantly though, Transnational
Entrepreneurs act as facilitators within this networks. Our next work
therefore is to characterize the frequency and the nature of
Transnational Diffusion as compared to other forms of diffusion across
networks and borders.

From @FactoryBerlin, 10,000 followers were collected. Of these 10,000,
419 were selected as interesting candidates. What made these users
candidates was that they were based in Berlin, and that they had
a valid friend:follower ratio. A valid friend:follower ratio means
that there was a disparity between the ratio of friends to followers
of no greater than 50\%. If there was a disparity greater than that,
there was an increased liklihood that the account was mass following
users in an attempt to grow their network (hoping that users will follow
back).

Of those 419 candidates, only 56 were selected as potential Transnational
Entrpreneuers. From those 419 candidates, a sample of 200 of their friends
was collected. Then, an analysis was performed to see if their ego-centric
friend network was composed of 2 or more sub-networks based in two locales.
The subnetworks could not vary in size by more than 25\%.

The first takeaway is that Transnational Entrepreneurs are exceedingly
rare. Of the 419 users that qualified for the initial analysis only
13.37\% were potentially Transnational Entrepreneurs.

From those 56 final Transnational Entrepreneur candidates, their
ego-centric network including their friends and followers (limit 200
for each) were extracted. Following this extraction, their friends and
followers were evaluated to make sure that they are not bots, and that
they were not spammers. This was checked in a number of ways, if they
were verified by twitter, if they had a follower:friend ratio within
bounds of 75\%, if they had over 50 tweets, and if they were not just
retweeting the same information over 50\% of the time, they qualified.

From these Transnational Ego-Centric networks, we collected a total of
2412594 tweets for analysis.
